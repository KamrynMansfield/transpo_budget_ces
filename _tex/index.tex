% Options for packages loaded elsewhere
% Options for packages loaded elsewhere
\PassOptionsToPackage{unicode}{hyperref}
\PassOptionsToPackage{hyphens}{url}
\PassOptionsToPackage{dvipsnames,svgnames,x11names}{xcolor}
%
\documentclass[
]{agujournal2019}
\usepackage{xcolor}
\usepackage{amsmath,amssymb}
\setcounter{secnumdepth}{5}
\usepackage{iftex}
\ifPDFTeX
  \usepackage[T1]{fontenc}
  \usepackage[utf8]{inputenc}
  \usepackage{textcomp} % provide euro and other symbols
\else % if luatex or xetex
  \usepackage{unicode-math} % this also loads fontspec
  \defaultfontfeatures{Scale=MatchLowercase}
  \defaultfontfeatures[\rmfamily]{Ligatures=TeX,Scale=1}
\fi
\usepackage{lmodern}
\ifPDFTeX\else
  % xetex/luatex font selection
\fi
% Use upquote if available, for straight quotes in verbatim environments
\IfFileExists{upquote.sty}{\usepackage{upquote}}{}
\IfFileExists{microtype.sty}{% use microtype if available
  \usepackage[]{microtype}
  \UseMicrotypeSet[protrusion]{basicmath} % disable protrusion for tt fonts
}{}
\makeatletter
\@ifundefined{KOMAClassName}{% if non-KOMA class
  \IfFileExists{parskip.sty}{%
    \usepackage{parskip}
  }{% else
    \setlength{\parindent}{0pt}
    \setlength{\parskip}{6pt plus 2pt minus 1pt}}
}{% if KOMA class
  \KOMAoptions{parskip=half}}
\makeatother
% Make \paragraph and \subparagraph free-standing
\makeatletter
\ifx\paragraph\undefined\else
  \let\oldparagraph\paragraph
  \renewcommand{\paragraph}{
    \@ifstar
      \xxxParagraphStar
      \xxxParagraphNoStar
  }
  \newcommand{\xxxParagraphStar}[1]{\oldparagraph*{#1}\mbox{}}
  \newcommand{\xxxParagraphNoStar}[1]{\oldparagraph{#1}\mbox{}}
\fi
\ifx\subparagraph\undefined\else
  \let\oldsubparagraph\subparagraph
  \renewcommand{\subparagraph}{
    \@ifstar
      \xxxSubParagraphStar
      \xxxSubParagraphNoStar
  }
  \newcommand{\xxxSubParagraphStar}[1]{\oldsubparagraph*{#1}\mbox{}}
  \newcommand{\xxxSubParagraphNoStar}[1]{\oldsubparagraph{#1}\mbox{}}
\fi
\makeatother


\usepackage{longtable,booktabs,array}
\usepackage{calc} % for calculating minipage widths
% Correct order of tables after \paragraph or \subparagraph
\usepackage{etoolbox}
\makeatletter
\patchcmd\longtable{\par}{\if@noskipsec\mbox{}\fi\par}{}{}
\makeatother
% Allow footnotes in longtable head/foot
\IfFileExists{footnotehyper.sty}{\usepackage{footnotehyper}}{\usepackage{footnote}}
\makesavenoteenv{longtable}
\usepackage{graphicx}
\makeatletter
\newsavebox\pandoc@box
\newcommand*\pandocbounded[1]{% scales image to fit in text height/width
  \sbox\pandoc@box{#1}%
  \Gscale@div\@tempa{\textheight}{\dimexpr\ht\pandoc@box+\dp\pandoc@box\relax}%
  \Gscale@div\@tempb{\linewidth}{\wd\pandoc@box}%
  \ifdim\@tempb\p@<\@tempa\p@\let\@tempa\@tempb\fi% select the smaller of both
  \ifdim\@tempa\p@<\p@\scalebox{\@tempa}{\usebox\pandoc@box}%
  \else\usebox{\pandoc@box}%
  \fi%
}
% Set default figure placement to htbp
\def\fps@figure{htbp}
\makeatother


% definitions for citeproc citations
\NewDocumentCommand\citeproctext{}{}
\NewDocumentCommand\citeproc{mm}{%
  \begingroup\def\citeproctext{#2}\cite{#1}\endgroup}
\makeatletter
 % allow citations to break across lines
 \let\@cite@ofmt\@firstofone
 % avoid brackets around text for \cite:
 \def\@biblabel#1{}
 \def\@cite#1#2{{#1\if@tempswa , #2\fi}}
\makeatother
\newlength{\cslhangindent}
\setlength{\cslhangindent}{1.5em}
\newlength{\csllabelwidth}
\setlength{\csllabelwidth}{3em}
\newenvironment{CSLReferences}[2] % #1 hanging-indent, #2 entry-spacing
 {\begin{list}{}{%
  \setlength{\itemindent}{0pt}
  \setlength{\leftmargin}{0pt}
  \setlength{\parsep}{0pt}
  % turn on hanging indent if param 1 is 1
  \ifodd #1
   \setlength{\leftmargin}{\cslhangindent}
   \setlength{\itemindent}{-1\cslhangindent}
  \fi
  % set entry spacing
  \setlength{\itemsep}{#2\baselineskip}}}
 {\end{list}}
\usepackage{calc}
\newcommand{\CSLBlock}[1]{\hfill\break\parbox[t]{\linewidth}{\strut\ignorespaces#1\strut}}
\newcommand{\CSLLeftMargin}[1]{\parbox[t]{\csllabelwidth}{\strut#1\strut}}
\newcommand{\CSLRightInline}[1]{\parbox[t]{\linewidth - \csllabelwidth}{\strut#1\strut}}
\newcommand{\CSLIndent}[1]{\hspace{\cslhangindent}#1}



\setlength{\emergencystretch}{3em} % prevent overfull lines

\providecommand{\tightlist}{%
  \setlength{\itemsep}{0pt}\setlength{\parskip}{0pt}}



 


\usepackage{url} %this package should fix any errors with URLs in refs.
\usepackage{lineno}
\usepackage[inline]{trackchanges} %for better track changes. finalnew option will compile document with changes incorporated.
\usepackage{soul}
\linenumbers
\makeatletter
\@ifpackageloaded{caption}{}{\usepackage{caption}}
\AtBeginDocument{%
\ifdefined\contentsname
  \renewcommand*\contentsname{Table of contents}
\else
  \newcommand\contentsname{Table of contents}
\fi
\ifdefined\listfigurename
  \renewcommand*\listfigurename{List of Figures}
\else
  \newcommand\listfigurename{List of Figures}
\fi
\ifdefined\listtablename
  \renewcommand*\listtablename{List of Tables}
\else
  \newcommand\listtablename{List of Tables}
\fi
\ifdefined\figurename
  \renewcommand*\figurename{Figure}
\else
  \newcommand\figurename{Figure}
\fi
\ifdefined\tablename
  \renewcommand*\tablename{Table}
\else
  \newcommand\tablename{Table}
\fi
}
\@ifpackageloaded{float}{}{\usepackage{float}}
\floatstyle{ruled}
\@ifundefined{c@chapter}{\newfloat{codelisting}{h}{lop}}{\newfloat{codelisting}{h}{lop}[chapter]}
\floatname{codelisting}{Listing}
\newcommand*\listoflistings{\listof{codelisting}{List of Listings}}
\makeatother
\makeatletter
\makeatother
\makeatletter
\@ifpackageloaded{caption}{}{\usepackage{caption}}
\@ifpackageloaded{subcaption}{}{\usepackage{subcaption}}
\makeatother
\usepackage{bookmark}
\IfFileExists{xurl.sty}{\usepackage{xurl}}{} % add URL line breaks if available
\urlstyle{same}
\hypersetup{
  pdftitle={Unpacking Household Budgeting Strategies through a Transportation Lens},
  pdfauthor={Kamryn Mansfield; Katie Asmussen},
  pdfkeywords={Consumer Expenditure Survey, Transportation
Spending, Family Spending, Budget},
  colorlinks=true,
  linkcolor={blue},
  filecolor={Maroon},
  citecolor={Blue},
  urlcolor={Blue},
  pdfcreator={LaTeX via pandoc}}



\draftfalse

\begin{document}
\title{Unpacking Household Budgeting Strategies through a Transportation
Lens}

\authors{Kamryn Mansfield\affil{1}, Katie Asmussen\affil{2}}
\affiliation{1}{University of Tennessee, }\affiliation{2}{University of
Tenenessee, }
\correspondingauthor{Kamryn Mansfield}{kmansfi4@vols.utk.edu}


\begin{abstract}
This is where we will put our abstract.
\end{abstract}

\section*{Plain Language Summary}
This is a plain language summary




\section{Introduction}\label{introduction}

Households juggle how to allocate their budgets: whether to invest in a
reliable car, pay for quality childcare, secure housing in a good school
district, or set money aside for leisure. These everyday choices shape
how families live and move, reflecting the trade-offs they make to
balance competing priorities. Transportation often sits at the center of
these decisions, not only because it can be a significant expense, but
also because choosing to buy and maintain a car versus relying on public
transit represents a long-term commitment and a broader lifestyle
choice. Its relative weight compared to housing, childcare, and other
spending varies widely across families. The relationship between
household budgeting and mobility is shaped not only by causal direction
but also by how families prioritize and weight different needs. On one
hand, mobility resources such as car ownership can structure the
household budget: households with no or only one vehicle may spend far
less on transportation, freeing up income for other essential or
discretionary categories. On the other hand, underlying family
structures and preferences can drive budget allocation choipces that, in
turn, shape transportation behavior. Larger families may prioritize
childcare or invest in higher-quality housing in areas with better
schools, limiting what remains for transportation. Others may emphasize
frugality across all categories or deliberately substitute toward
lower-cost transit options. Understanding both the direction of
influence and the weight assigned to different budget categories is
critical for transportation planning and policy, as these dynamics
reveal how families navigate competing priorities under varying
demographic and mobility contexts.

The purpose of this research is to explore how household budgets are
structured around transportation decisions and how this impacts other
spending categories. Using the Consumer Expenditure Survey (CEX), we
will perform a Latent Class Analysis (LCA) to find groupings based on a
household's transportation expenses. These groupings can help us find
groups of spenders with similar patterns to help us predict
transportation expenses based on the household's characteristics.

\section{Literature Review}\label{literature-review}

\textsubscript{Source:
\href{https://KamrynMansfield.github.io/transpo_budget_ces/index.qmd.html}{Article
Notebook}}

Table~\ref{tbl-lit-summary} summarizes the literature that was reviewed
for this study. As seen in the table, there were a total of 35 studies
reviewed with 19 studies that use the consumer expenditure survey.


\begin{longtable}[]{@{}
  >{\raggedright\arraybackslash}p{(\linewidth - 6\tabcolsep) * \real{0.1556}}
  >{\raggedright\arraybackslash}p{(\linewidth - 6\tabcolsep) * \real{0.4370}}
  >{\raggedright\arraybackslash}p{(\linewidth - 6\tabcolsep) * \real{0.1111}}
  >{\raggedright\arraybackslash}p{(\linewidth - 6\tabcolsep) * \real{0.2815}}@{}}

\caption{\label{tbl-lit-summary}Summary of the Literature}

\tabularnewline

\toprule\noalign{}
\begin{minipage}[b]{\linewidth}\raggedright
Title
\end{minipage} & \begin{minipage}[b]{\linewidth}\raggedright
Citation
\end{minipage} & \begin{minipage}[b]{\linewidth}\raggedright
Dataset
\end{minipage} & \begin{minipage}[b]{\linewidth}\raggedright
Findings
\end{minipage} \\
\midrule\noalign{}
\endhead
\bottomrule\noalign{}
\endlastfoot
A family expenditure perspective on transport planning: Australian
evidence in context &
Morris \& Wigan (1979)
& Australian Expenditure Survey & transport takes about o17 percent of
expenditures, at least part of the consumption expenditure by low income
families is financed from savings, loans or other sourCEX besides
``income''. \\
Variable Lifespan and the Intertemporal Elasticity of Consumption &
Skinner (1985)
& CEX & Changes in inflation will prompt changes in consumer
expenditures \\
A sample selection model for prepared food expenditures &
Nayga (1998)
& CEX & Most of the variables analysed significantly affect prepared
food expenditures. For example, results suggest that frozen meals
expenditures are higher for households without children, for smaller
households, and for households headed by a non-white individual \\
Transportation Costs and Economic Opportunity Among the Poor &
Blumenberg (2003)
& CEX & Cost comparisons fall short of finding if transportation costs
are a barrier for economic opportunity among the poor \\
Analysis of Variations in Vehicle Ownership Expenditures &
Thakuriah \& Liao (2005)
& CEX & For vehicle-owning households, of every additional dollar that
households spend, 18 cents is spent on vehicles after controlling for
socioeconomic, demographic, life cycle, and other factors relating to
households. \\
Leisure Travel Expenditure Patterns by Family Life Cycle Stages &
Hong et al. (2005)
& CEX & Marrieds without children are more likely to spend on leisure
travel than singles, whereas single parents and solitary survivors are
less likely to spend on leisure travel than singles. \\
Transportation Expenditures and Ability to Pay: Evidence from Consumer
Expenditure Survey &
Thakuriah (Vonu) \& Liao (2006)
& CEX & Transportation expenditures made by households are better
explained by permanent income levels of households than by annual
incomes. \\
The Effects of Ethnic Identity on Household Budget Allocation to Status
Conveying Goods &
Fontes \& Fan (2006)
& CEX & Asian Americans allocate more of their budget to housing,
African Americans allocate more of their budget to apparel, and
Hispanics allocate more of their budget to both housing and apparel, but
to a lesser extent than Asian Americans with respect to housing and
African Americans with respect to apparel. \\
Do Transportation and Communications Tend to be Substitutes,
Complements, or Neither?: U.S. Consumer Expenditures Perspective,
1984--2002 &
Choo et al. (2007)
& CEX & New tech doesn't substitue Personal Vehicle travel probably \\
An analysis of the determinants of children's weekend physical activity
participation &
Rachel B. Copperman \& Bhat (2007a)
& San Francisco Bay Area Travel Survey & The ``number of children''
variable suggests an overall higher likelihood of participation in
utilitarian active travel among households with many children relative
to households with few children \\
An analysis of the social context of children's weekend discretionary
activity participation &
Sener \& Bhat (2007)
& Panel Study of Income Dynamics (PSID) & male children more likely to
participate with their fathers than female children, African-American
children less likely to participate in health-enhancing active
recreation pursuits \\
An Exploratory Analysis of Children's Daily Time-Use and Activity
Patterns Using the Child Development Supplement (CDS) to the US Panel
Study of Income Dynamics (PSID) &
Rachel B. Copperman \& Bhat (2007b)
& Panel Study of Income Dynamics (PSID) & The age of childrend has an
effect on the types of activities they pursue \\
Estimating Transportation Costs by Characteristics of Neighborhood and
Household &
Haas et al. (2008)
& CEX and others & Their model can help in travel demand modeling \\
An analysis of children's leisure activity engagement: Examining the day
of week, location, physical activity level, and fixity dimensions &
Sener et al. (2008)
& Panel Study of Income Dynamics (PSID) & Children in households with
parents who are employed, higher income, or higher education were found
to participate in structured outdoor activities at higher rates. \\
A comprehensive analysis of household transportation expenditures
relative to other goods and serviCEX: An application to United States
consumer expenditure data &
Ferdous et al. (2010)
& CEX & Adjustments are made with increased fuel priCEX \\
Measuring the Structureal Determinants of Urban Travel Demand &
Souche (2010)
& IUTP database & urban density and cost of transport mode were
statisticalliy significan in their model \\
An empirical analysis of children's after school out-of-home
activity-location engagement patterns and time allocation &
Paleti et al. (2011)
& Panel Study of Income Dynamics (PSID) & The results show that a wide
variety of demographic, attitudinal, environmental, and others'
activity-travel pattern characteristics impact children's after school
activity engagement patterns. \\
Is the Consumer Expenditure Survey Representative by Income? &
Sabelhaus et al. (2013)
& CEX & the highest income thresholds are underrepresented in the
survey \\
Relationship between Households' Housing and Transportation
Expenditures: Examination from Lifestyle Perspective &
Deka (2015)
& CEX & More housing costs = more transportation costs, people the take
transit spend less on transportation \\
An empirical investigation into the time-use and activity patterns of
dual-earner couples with and without young children &
Bernardo et al. (2015)
& ATUS & the presence of a child in dual-earner households not only
leads to a reduction in in-home non-work activity participation
(excluding child care activities) but also a substantially larger
decrease in out-of-home non-work activity participation (excluding child
care and shopping activities), \\
Expenditures on Children by Families, 2015 &
Lino et al. (2017)
& CEX & Many observations on the expenditures of children \\
Factors influencing travel mode choice among families with young
children (aged 0--4): A review of the literature &
McCarthy et al. (2017)
& Lit Review & many factores influence decisions about mode choice when
traveling with young children. \\
Relationships between Land Use, Transportation, Household Expenditures,
and Municipal Spending in Small Urban Areas &
Mattson \& Peterson (2019)
& CEX, Ammerican Community Survey & denser areas are more likely to use
transit to commute. People in single-family homes tend to spend more
money on transportation \\
An environment-people interactions framework for analysing children's
extra-curricular activities and active transport &
Leung et al. (2019)
& Survey in Hong Kong Schools & childrens activities can differ a lot
based on neighborhood enfironment and family sociodemographic
background. \\
Relationships between density, transit, and household expenditures in
small urban areas &
Mattson (2020)
& CEX, Ammerican Community Survey & single family homes spend more on
transportation, higher income is correlated with higher tranportation
costs. \\
An Updated Estimation Model of the Cost of Raising Children in Texas &
Osborne et al. (2021)
& NHTS, CEX & Regardless of the method of calculation, we find that it
is nearly impossible for two minimum wage earners to meet the basic
costs of raising children in Texas, especially when child care is
included \\
Parental Investments of Money for White, Black, and Hispanic Children in
the United States &
Hastings (2022)
& CEX & . Both sociodemographic and economic factors play a substantial
role in these differenCEX, and the racial and ethnic gaps in parental
investments of money are nearly eliminated when both are accounted
for \\
The role of household modality style in first and last mile travel mode
choice &
Lu et al. (2022)
& Southeast Queensland Travel Survey (SEQTS) & joint travel contributes
least to modal shift from car to active transport when there is improved
infrastructure of trains and things \\
A review of the socio-demographic characteristics affecting the demand
for different car-sharing operational schemes &
Amirnazmiafshar \& Diana (2022)
& NA & There are lots of factors that might affect people's willingness
to use car sharing \\
Estimating Expenditures on Children by Families in Canada, 2014 to 2017
&
Duncan et al. (2023)
& Survey of Household Spending (SHS) & The more income, the more
spending on kids. \\
A New Approach to Understanding the Impact of Automobile Ownership on
Transportation Equity &
Molloy et al. (2024)
& CEX & ``Captive Riders'' have less spending allocated to
transportation than captive drivers. \\
Transportation Statistics Annual Report 2024 &
Bureau of Transportation Statistics (2024)
& CEX and others & Lots of summaries \\
A STUDY OF SPENDING, SAVING AND INVESTMENT PATTERNS OF MARRIED COUPLES
WITH CHILDREN(NON-DINK) IN MUMBAI &
Hargunani et al. (2024)
& Data from Mumbai & The data analysis reveals distinct spending,
saving, and investment patterns among married couples, with a clear
prioritization towards ensuring the well-being and future security of
their families.'' \\
Subsidizing Car Ownership for Low-Income Individuals and Households &
Klein (2024)
& Personal Survey Created by Dr.~Klein & Having a car gave people more
opportunities than before, and they usually had more time to spend with
the family. At the beginning and end of the day. \\
Investigating the effects of ridesourcing on the dynamics of household
car ownership &
Bilgin et al. (2025)
& United Kingdom Household Longitudinal Study & Suggests that households
are less likely to qcquire a car in the presence of ridesourcing, but
car disposal is mainly driven by household compositions and residential
relocation factors. \\

\end{longtable}

\textsubscript{Source:
\href{https://KamrynMansfield.github.io/transpo_budget_ces/notebooks/lit_review_table-preview.html\#cell-tbl-lit-summary}{Lit
Review Table}}

The literature relating to this study has been classified into four
groups: (1) Family Choices and Activity Patterns, (2) Family
Transportation Choices, (3) Family Spending and Budgets, and (4) Family
Transportation Budgets. The following sections describe the relevant
findings from literature in each of these groups.

\subsection{Family Choices and Activity
Patterns}\label{family-choices-and-activity-patterns}

There have been many studies done on the choices and activities of
families (Rachel B. Copperman \& Bhat, 2007b; Leung et al., 2019; Sener
et al., 2008; see Sener \& Bhat, 2007). These studies often focus on the
activities choices of households and children.

Paleti et al. (2011) performed a study where they wanted to characterize
the activity patterns of children after school. Their data were gathered
from the Child Development Supplement to the Panel Study of Income
Dynamics which has household demographics and time-use diaries for
children. They looked at travel patterns using combinations of three
activity-travel scenarios: staying at school, going home from school,
and going somewhere else after school. They further identified specific
after-school activities (e.g.~Organized activities at school, recreation
at the home of someone else, meals at restaurants, etc.) to use in a
multiple discrete-continuous extreme value (MDCEV) model. The MDCEV is a
type of discrete choice model that works when multiple options can be
chosen, and was used to find predictors of children's participation in
the different after school activities. In their analysis, they found
that 57.7\% of children in the survey participated in at least one
out-of-home activity after school. They also found that children's
activities were connected to household income, family dynamics,
environment, and other things. For example, children in households with
higher income were more likely to participate in activities after
school. Children with no siblings along with children having a working
primary caregiver were more likely to stay at school or go somewhere
besides home directly after school. Children living close to a large
city were less likely to go somewhere after school, go home, and then go
back out. The findings of this study show the variety of factors that
might affect a family's activity, and therefore transportation,
patterns.

Another study on family choices was done by Bernardo et al. (2015). They
used the American Time Use Survey and a Multiple Discrete Continuous
Nested Extreme Value (MDCNEV) model to examine the activities of
duel-earner households. The variables they used relate to household
demographics, respondent demographics, couple characteristics, and day
of the week. Findings indicated that women are more likely to
participate in out-of-home maintenance, shopping, and social activities
than men. They also found that respondents with higher education and
with children are more likely to work from home. One key finding of this
study is that couples with children are much less likely to participate
in out-of-home, non-work activities.

\subsection{Family Transportation
Choices}\label{family-transportation-choices}

Among the studies on family choices is a group of studies that focus on
family transportation choices (Amirnazmiafshar \& Diana, 2022; Rachel B.
Copperman \& Bhat, 2007a; Lu et al., 2022; Souche, 2010). These studies
look at the connection between family mobility and family decisions.

McCarthy et al. (2017) is a literature review with some good findings,
but I don't know if I should site the literature review or if I should
find individual papers from the review to talk about.

A unique study to understand the effects car ownership has on household
decisions was done by Nicholas Klein (2024). In order to understand how
access to a car can effect a family in the United States, he interviewed
30 people in Maryland and Virginia who received a subsidized car. Two
main findings of this study relate to travel behavior changes and access
to opportunities. The people interviewed generally changed their travel
behavior in similar ways after receiving a car. Before receiving the
car, they would rely on public transit and others for transportation,
but after receiving a car, they made many trips in their own cars,
including some trips that they had to forgo before having a car. Another
general conclusion Klein makes is that people had more access to
opportunities after receiving a car. They had easier access to more
potential jobs, but some also mentioned the ability to get more hours at
the their current jobs. With less reliance on public transit, many
respondents spent more time with their families at the beginning and end
of the day.

Another study interested in car ownership was done by
@bilgin\_investigating\_2025. They analyzed car ownership across
multiple years using the United Kingdom Household Longitudinal Study
dataset to see if ridesourcing availability affects car ownership. They
used two fixed effects logit models: one to model the effect of
ridesourcing on the decision to increase the number of cars in the
household and the other to model the effect of ridesourcing on the
decision to decrease the number of cars in the household. Their results
suggested that households with more than one car are more likely to get
rid of a car and less likely to add a car compared to households with
one car. Even with this tendency, their models did not show a strong
connection between the presence of ride sourcing and changes in car
ownership. They concluded that changes in household composition have a
stronger impact on the change in number of cars of a household.

\subsection{Family Spending and
Budgets}\label{family-spending-and-budgets}

Another set of studies focuses on household budgets and household
spending patterns (Fontes \& Fan, 2006; Nayga, 1998; Sabelhaus et al.,
2013; Skinner, 1985). Many of the studies reviewed had an emphasis on
the budgets related to raising children. Hargunani et al. (2024)
analyzed family spending patterns in Mumbai and concluded that many
families focus their expenditures on the current and future wellbeing of
their children. This is evidenced by money spent on basic necessities
and setting aside money for the future.

The United States Department of Agriculture (USDA) has produced reports
that use the CEX to specifically analyze the costs of raising a child in
the United States. The most recent report (Lino et al., 2017) found the
top expenditure for married-couple families with two children to be
housing. The rankings of other expenditures were different depending on
the age of children, but food, child care/education, and transportation
were always the next highest expenditures on children. Similar to the
USDA report on the cost of raising children, Osborne et al. (2021)
modeled the cost of raising children in Texas by following similar
methodologies but using Texas-specific data for housing and childcare
costs. They looked not only at married-couple families, but also at
single-parent households and duel households where children spend time
with both parents in different locations. They found differing
expenditures on children among the different family make-ups and among
different incomes.

Other studies with similar analyses have had similar findings.
@hastings\_parental\_2022 used the CEX to compare expenditures between
different racial and ethnic groups. When controlling for both family
characteristics and income, he found that there was not a significant
difference in total expenditures on children among racial and ethnic
groups. This suggests that income and family characteristics play a
larger role in family budgeting than race and ethnicity. Duncan et al.
(2023) performed a study in Canada using the country's Survey of
Household Spending (SHS) to analyze family expenditures. They found
similar results as previously mentioned studies. Different income groups
had different amounts allocated to children, but housing was always the
highest expenditure with food, child care/education, and transportation
being the next highest expenditures.

\subsection{Family Transportation Expenses and
Budgets}\label{family-transportation-expenses-and-budgets}

There have been many studies on family budgets and transportation
expenses (Blumenberg, 2003; Choo et al., 2007; Ferdous et al., 2010;
Haas et al., 2008; Hong et al., 2005; Morris \& Wigan, 1979; Thakuriah
(Vonu) \& Liao, 2006).

One study focused on transportation budgets was done by Thakuriah \&
Liao (2005). Using CEX data, they made multiple models to analyze the
expenditures related to vehicle ownership of households in the United
States. In each model, they used a variety of variables (income,
household demographics, spatial factors, economic factors, and family
condition factors) to predict the amount of money a household spends on
vehicles. Their model results indicate 18 percent of additional
household expenditures is a vehicle expense. They results also indicate
many factors influence household vehicle expenses. The models showed
that homeowners spend more on vehicle expenses. They also showed that
vehicle expenses are connected with the sex of the head of household and
the number of people in the household.

In a study by Deka (2015), they used two years of the CEX to see the
connection between housing expenses and transportation expenses. They
created three least squares models: one to describe expenditures in
dollar amounts, one to describe expenditures as percentages of income,
and one to describe expenditures as shares of total household
expenditures. They found a positive association between transportation
expenses and housing expenses. Similarly, share of income on
transportation was positively affected by share of income on housing,
but there was no significant evidence showing the share of income on
housing being affected by share of income on transportation. Both
housing and transportation expenditures were found to be higher for
those living in single-family detached homes and lower for those living
in older homes.

Mattson (2020) Mattson \& Peterson (2019) - single family homes spend
more on transportation, higher income is correlated with higher
transportation costs. - denser areas are more likely to use transit to
commute. People in single-family homes tend to spend more money on
transportation

Mattson \& Peterson (2019) looked at how density, land use,
transportation, and household expenditures are related. They developed a
regression model using CEX data where transportation expenditures were
estimated using type of housing, population, family characteristics, and
other factors. Similar to other studies, they found that income is
positively associated with transportation expenditures. Their model
showed single-family homes to have higher transportation expenditures
than other types of dwellings. Using data from the U.S. Census Bureau's
Annual Survey of State and Local Government Finances, they also created
models to explore how land use might affect municipal spending. In these
models, increase in density was associated with a decrease in many per
capita operational costs, including fire protection, streets and
highways, parks and recreation, and others.

A similar study was published by Mattson (2020) where he modeled
transportation expenditures using CEX data. The results showed, like the
previous publication, that single-family dwellings spend more on
transportation compared to other types. He also found that larger
household sizes and newer homes contribute to higher transportation
spending.

Molloy et al. (2024) - ``Captive Riders'' have less spending allocated
to transportation than captive drivers.

In their study, Molloy et al. (2024) look at CEX data through after
proposing new framework to look at transit users and vehicle users. In
past studies, three categories have been used to analyze transportation
users: ``drivers'' are those who own a car, ``captive riders'' are those
who use transit because they ca not afford a car, and ``choice riders''
are those who can afford a car but do not. Instead of having one
grouping for those that own cars, Molloy et al. (2024) propose splitting
that category into ``captive drivers'' and ``choice drivers''. Captive
drivers would be those in the population that are low income but have a
car representing people with less transportation freedom. They analyzed
the CEX with this new way of classifying transportation users and found
that captive drivers carry the most transportation burden. The
transportation expenditures of captive drivers average more than 16\% of
total household expenditures while the transportation expenditures of
captive riders was only around 7\% of total expenditures.

Bureau of Transportation Statistics (2024) - Chapter 2 has a whole
section where they analyze household expenditures gettting much of their
data from the CEX.

\section{References}\label{references}

\phantomsection\label{refs}
\begin{CSLReferences}{1}{0}
\vspace{1em}

\bibitem[\citeproctext]{ref-amirnazmiafshar_review_2022}
Amirnazmiafshar, E., \& Diana, M. (2022). A review of the
socio-demographic characteristics affecting the demand for different
car-sharing operational schemes. \emph{Transportation Research
Interdisciplinary Perspectives}, \emph{14}, 100616.
\url{https://doi.org/10.1016/j.trip.2022.100616}

\bibitem[\citeproctext]{ref-bernardo_empirical_2015}
Bernardo, C., Paleti, R., Hoklas, M., \& Bhat, C. (2015). An empirical
investigation into the time-use and activity patterns of dual-earner
couples with and without young children. \emph{Transportation Research
Part A: Policy and Practice}, \emph{76}, 71--91.
\url{https://doi.org/10.1016/j.tra.2014.12.006}

\bibitem[\citeproctext]{ref-bilgin_investigating_2025}
Bilgin, P., Mattioli, G., Morgan, M., \& Wadud, Z. (2025). Investigating
the effects of ridesourcing on the dynamics of household car ownership.
\emph{Transportation Research Part D: Transport and Environment},
\emph{146}, 104886. \url{https://doi.org/10.1016/j.trd.2025.104886}

\bibitem[\citeproctext]{ref-blumenberg_transportation_2003}
Blumenberg, E. (2003). Transportation {Costs} and {Economic}
{Opportunity} {Among} the {Poor}.

\bibitem[\citeproctext]{ref-bureau_of_transportation_statistics_transportation_2024}
Bureau of Transportation Statistics. (2024). \emph{Transportation
{Statistics} {Annual} {Report} 2024} (pp. 219 pages, 35.3 Megabytes).
Bureau of Transportation Statistics.
\url{https://doi.org/10.21949/E0KQ-GF72}

\bibitem[\citeproctext]{ref-choo_transportation_2007}
Choo, S., Lee, T., \& Mokhtarian, P. L. (2007). Do {Transportation} and
{Communications} {Tend} to be {Substitutes}, {Complements}, or
{Neither}?: {U}.{S}. {Consumer} {Expenditures} {Perspective},
1984--2002. \emph{Transportation Research Record}, \emph{2010}(1),
121--132. \url{https://doi.org/10.3141/2010-14}

\bibitem[\citeproctext]{ref-copperman_analysis_2007}
Copperman, Rachel B., \& Bhat, C. R. (2007a). An analysis of the
determinants of children's weekend physical activity participation.
\emph{Transportation}, \emph{34}(1), 67--87.
\url{https://doi.org/10.1007/s11116-006-0005-5}

\bibitem[\citeproctext]{ref-copperman_exploratory_2007}
Copperman, Rachel B., \& Bhat, C. R. (2007b). An {Exploratory}
{Analysis} of {Children}'s {Daily} {Time}-{Use} and {Activity}
{Patterns} {Using} the {Child} {Development} {Supplement} ({CDS}) to the
{US} {Panel} {Study} of {Income} {Dynamics} ({PSID}).

\bibitem[\citeproctext]{ref-deka_relationship_2015}
Deka, D. (2015). Relationship between {Households}' {Housing} and
{Transportation} {Expenditures}: {Examination} from {Lifestyle}
{Perspective}. \emph{Transportation Research Record}, \emph{2531}(1),
26--35. \url{https://doi.org/10.3141/2531-04}

\bibitem[\citeproctext]{ref-duncan_estimating_2023}
Duncan, K. A., Frank, K., \& Guèvremont, A. (2023). Estimating
{Expenditures} on {Children} by {Families} in {Canada}, 2014 to 2017.
\url{https://doi.org/10.25318/11F0019M2023007-ENG}

\bibitem[\citeproctext]{ref-ferdous_comprehensive_2010}
Ferdous, N., Pinjari, A. R., Bhat, C. R., \& Pendyala, R. M. (2010). A
comprehensive analysis of household transportation expenditures relative
to other goods and services: An application to {United} {States}
consumer expenditure data. \emph{Transportation}, \emph{37}(3),
363--390. \url{https://doi.org/10.1007/s11116-010-9264-2}

\bibitem[\citeproctext]{ref-fontes_effects_2006}
Fontes, A., \& Fan, J. (2006). The {Effects} of {Ethnic} {Identity} on
{Household} {Budget} {Allocation} to {Status} {Conveying} {Goods}.
\emph{Journal of Family and Economic Issues}, \emph{27}, 643--663.
\url{https://doi.org/10.1007/s10834-006-9031-x}

\bibitem[\citeproctext]{ref-haas_estimating_2008}
Haas, P. M., Makarewicz, C., Benedict, A., \& Bernstein, S. (2008).
Estimating {Transportation} {Costs} by {Characteristics} of
{Neighborhood} and {Household}. \emph{Transportation Research Record},
\emph{2077}(1), 62--70. \url{https://doi.org/10.3141/2077-09}

\bibitem[\citeproctext]{ref-hargunani_study_2024}
Hargunani, C., Vernekar, S., \& Vernekar, S. (2024). A {STUDY} {OF}
{SPENDING}, {SAVING} {AND} {INVESTMENT} {PATTERNS} {OF} {MARRIED}
{COUPLES} {WITH} {CHILDREN}({NON}-{DINK}) {IN} {MUMBAI}, \emph{20}(1).

\bibitem[\citeproctext]{ref-hastings_parental_2022}
Hastings, O. P. (2022). Parental {Investments} of {Money} for {White},
{Black}, and {Hispanic} {Children} in the {United} {States}.
\emph{Socius: Sociological Research for a Dynamic World}, \emph{8},
23780231221103054. \url{https://doi.org/10.1177/23780231221103054}

\bibitem[\citeproctext]{ref-hong_leisure_2005}
Hong, G.-S., Fan, J. X., Palmer, L., \& Bhargava, V. (2005). Leisure
{Travel} {Expenditure} {Patterns} by {Family} {Life} {Cycle} {Stages}.
\emph{Journal of Travel \& Tourism Marketing}, \emph{18}(2), 15--30.
\url{https://doi.org/10.1300/J073v18n02_02}

\bibitem[\citeproctext]{ref-klein_subsidizing_2024}
Klein, N. J. (2024). Subsidizing {Car} {Ownership} for {Low}-{Income}
{Individuals} and {Households}. \emph{Journal of Planning Education and
Research}, \emph{44}(1), 165--177.
\url{https://doi.org/10.1177/0739456X20950428}

\bibitem[\citeproctext]{ref-leung_environment-people_2019}
Leung, K. Y. K., Astroza, S., Loo, B. P. Y., \& Bhat, C. R. (2019). An
environment-people interactions framework for analysing children's
extra-curricular activities and active transport. \emph{Journal of
Transport Geography}, \emph{74}, 341--358.
\url{https://doi.org/10.1016/j.jtrangeo.2018.12.015}

\bibitem[\citeproctext]{ref-lino_expenditures_2017}
Lino, M., Kuczynski, K., Rodriguez, N., \& Schap, T. (2017).
\emph{Expenditures on {Children} by {Families}, 2015}. United States
Department of Agriculture. \url{https://doi.org/10.22004/ag.econ.327257}

\bibitem[\citeproctext]{ref-lu_role_2022}
Lu, Y., Prato, C. G., Sipe, N., Kimpton, A., \& Corcoran, J. (2022). The
role of household modality style in first and last mile travel mode
choice. \emph{Transportation Research Part A: Policy and Practice},
\emph{158}, 95--109. \url{https://doi.org/10.1016/j.tra.2022.02.003}

\bibitem[\citeproctext]{ref-mattson_relationships_2020}
Mattson, J. (2020). Relationships between density, transit, and
household expenditures in small urban areas. \emph{Transportation
Research Interdisciplinary Perspectives}, \emph{8}, 100260.
\url{https://doi.org/10.1016/j.trip.2020.100260}

\bibitem[\citeproctext]{ref-mattson_relationships_2019}
Mattson, J., \& Peterson, D. (2019). Relationships between {Land} {Use},
{Transportation}, {Household} {Expenditures}, and {Municipal} {Spending}
in {Small} {Urban} {Areas}.

\bibitem[\citeproctext]{ref-mccarthy_factors_2017}
McCarthy, L., Delbosc, A., Currie, G., \& Molloy, A. (2017). Factors
influencing travel mode choice among families with young children (aged
0--4): A review of the literature. \emph{Transport Reviews},
\emph{37}(6), 767--781.
\url{https://doi.org/10.1080/01441647.2017.1354942}

\bibitem[\citeproctext]{ref-molloy_new_2024}
Molloy, Q., Garrick, N., \& Atkinson-Palombo, C. (2024). A {New}
{Approach} to {Understanding} the {Impact} of {Automobile} {Ownership}
on {Transportation} {Equity}. \emph{Transportation Research Record},
\emph{2678}(2), 366--376.
\url{https://doi.org/10.1177/03611981231174444}

\bibitem[\citeproctext]{ref-morris_family_1979}
Morris, J. M., \& Wigan, M. R. (1979). A family expenditure perspective
on transport planning: {Australian} evidence in context.
\emph{Transportation Research Part A: General}, \emph{13}(4), 249--285.
\url{https://doi.org/10.1016/0191-2607(79)90051-7}

\bibitem[\citeproctext]{ref-nayga_sample_1998}
Nayga, R. M. (1998). A sample selection model for prepared food
expenditures. \emph{Applied Economics}, \emph{30}(3), 345--352.
\url{https://doi.org/10.1080/000368498325868}

\bibitem[\citeproctext]{ref-osborne_updated_2021}
Osborne, C., Wu, E., \& Benson, K. (2021). \emph{An {Updated}
{Estimation} {Model} of the {Cost} of {Raising} {Children} in {Texas}}.

\bibitem[\citeproctext]{ref-paleti_empirical_2011}
Paleti, R., Copperman, R. B., \& Bhat, C. R. (2011). An empirical
analysis of children's after school out-of-home activity-location
engagement patterns and time allocation. \emph{Transportation},
\emph{38}(2), 273--303. \url{https://doi.org/10.1007/s11116-010-9300-2}

\bibitem[\citeproctext]{ref-sabelhaus_is_2013}
Sabelhaus, J., Johnson, D., Ash, S., Swanson, D., Garner, T., Greenlees,
J., \& Henderson, S. (2013). \emph{Is the {Consumer} {Expenditure}
{Survey} {Representative} by {Income}?} (No. w19589). National Bureau of
Economic Research. \url{https://doi.org/10.3386/w19589}

\bibitem[\citeproctext]{ref-sener_analysis_2007}
Sener, I. N., \& Bhat, C. R. (2007). An analysis of the social context
of children's weekend discretionary activity participation.
\emph{Transportation}, \emph{34}(6), 697--721.
\url{https://doi.org/10.1007/s11116-007-9125-9}

\bibitem[\citeproctext]{ref-sener_analysis_2008}
Sener, I. N., Copperman, R. B., Pendyala, R. M., \& Bhat, C. R. (2008).
An analysis of children's leisure activity engagement: Examining the day
of week, location, physical activity level, and fixity dimensions.
\emph{Transportation}, \emph{35}(5), 673--696.
\url{https://doi.org/10.1007/s11116-008-9173-9}

\bibitem[\citeproctext]{ref-skinner_variable_1985}
Skinner, J. (1985). Variable {Lifespan} and the {Intertemporal}
{Elasticity} of {Consumption}. \emph{The Review of Economics and
Statistics}, \emph{67}(4), 616--623.
\url{https://doi.org/10.2307/1924806}

\bibitem[\citeproctext]{ref-souche_measuring_2010}
Souche, S. (2010). Measuring the structural determinants of urban travel
demand. \emph{Transport Policy}, \emph{17}(3), 127--134.
\url{https://doi.org/10.1016/j.tranpol.2009.12.003}

\bibitem[\citeproctext]{ref-thakuriah_analysis_2005}
Thakuriah, P. (Vonu)., \& Liao, Y. (2005). Analysis of {Variations} in
{Vehicle} {Ownership} {Expenditures}. \emph{Transportation Research
Record: Journal of the Transportation Research Board}, \emph{1926}(1),
1--9. \url{https://doi.org/10.1177/0361198105192600101}

\bibitem[\citeproctext]{ref-thakuriah_vonu_transportation_2006}
Thakuriah (Vonu), P., \& Liao, Y. (2006). Transportation {Expenditures}
and {Ability} to {Pay}: {Evidence} from {Consumer} {Expenditure}
{Survey}. \emph{Transportation Research Record}, \emph{1985}(1),
257--265. \url{https://doi.org/10.1177/0361198106198500128}

\end{CSLReferences}




\end{document}
